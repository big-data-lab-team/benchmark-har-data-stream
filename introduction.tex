\section{Introduction}
\label{sec:introduction}
%Introduce the internet of thing.
From smart lights~\cite{smart-lamp-2011} to smart
cities~\cite{smart-cities}, the requirements of
the Internet of Things (IoT) are constantly
growing~\cite{iot-platform}. 
%Description of the IoT
The Internet of
Things is a concept in which a wide variety
of objects are connected to the Internet to
provide new services.  The
type of objects that can be connected could be a
watch\footnote{\url{https://www.apple.com/watch/}}
as well as a street light~\cite{smart-lamp-2011}
or a basic sensor. A sensor is a very small device
designed to detect events in its environment such
as temperature, humidity, or acceleration.  The
sensor is said wearable when it is incorporated
into clothing or worn as an implant. These objects often
embed a wireless communication system like wifi
or Bluetooth.  Finally, they can be
connected to Internet directly, with their own IP
address, or indirectly, through another connected
object such as a phone.

%Example of use
Connected objects can be used in various
situations.  For instance, a network of street
lights can be designed to keep a high feeling of
security while minimizing energy consumption.  On
the other hand, wearable sensors that gather data
about human activity, as described
in~\cite{recofit}, can be leveraged to improve
athletes performances and health. 
As a concrete example, the Motsai company develops
the Neblina, a wearable device as big as a coin
that incorporates nine motion sensors with 64KB
of main
memory\footnote{\url{http://docs.motsai.com/Neblina/Neblina_Module/V2/Datasheet.html}}.
This wearable sensor is designed to track and
analyze human motion.

%Describe the pipeline
The current pipelines that process data from
connected appliances are centralized, as
illustrated in~\cite{recofit}.  Data are produced
on a device, then transmitted to a centralized
cloud or a laptop where they are stored and turned
into a proper dataset. Then, the dataset is used
to extract useful information in an offline
manner.  The extraction is qualified as offline
because it starts once all data are collected. For
instance, when training a Machine learning model,
the data are gather then carefully cured to select
the best features. Then the model is trained and
tested on the data set.

%Explain the drawback of this pipeline
However, these data analysis pipelines need to be
updated as centralizing the raw data raises a
few issues.
%Privacy issue
For instance, in the case of smart homes, the
user would like the use of their data to respect
their privacy.
They may not agree to disclose their bedtime or
how long they usually stay in the restroom.
%Battery issue
In addition to this privacy concern, the
centralized approach diminishes the batteries
lifetime of the object because transmitting
information is the most energy-consuming action
for a connected
device~\cite{sensor-network-survey,
sensor-energy-model}.  
%NOTE: Keep the next three lines :D.
%\cite{sensor-energy-consumption} (conclusion, second paragraph) communication uses more energy.
%\cite{leach} and \cite{sensor-energy-model}(page 3, first column, check equations and values)
%\cite{sensor-network-survey} : "Since the sensor nodes are often inaccessible, the lifetime of a sensor network depends on the lifetime of the power resources of the nodes"

%Maintenance problems (not necesserly related to the pipeline.)
Let aside those pipeline related problems, the
spreading of inter-connected objects in wide
areas could cause maintenance issues. For instance,
a farmer having a network of sensors spread across
his land is unlikely to have time to change
broken sensors every two weeks. Given the fact
that sensors are more likely to exhaust their
battery than being shattered by an external
cause~\cite{sensor-network-survey}, the
maintenance matter would greatly benefit from
energy-efficient communication algorithms.

%Security issue
%Finally, users could
%also worry about the security of their new object.
%Recent events~\cite{pacemakers-hacked} have shown
%the fragility of the automated object security. A
%fragile security could lead to a paralyzed home
%and to unwillingly extracted data from the user.


To tackle these privacy and energy issues,
algorithms with small memory footprint were
developed to mine information directly on the
devices rather than in a centralized cloud.  These
algorithms were inspired of Big Data, in
particular, from data stream algorithms because it
is common for these fields to handle more data
than available resources can accommodate.
Designing such method allow transmitting only the
relevant data to the cluster. The survey
in~\cite{kejariwal2015} reviews the problems faced
by connected devices.
However,~\cite{kejariwal2015} did not cover the
problem of supervised learning which raises the
following question: Is it possible to design
classification algorithms for data streams on
connected devices? 

%Outline of the report
The next section will explain the main data stream
algorithms in more details and a few problems will
be described.  Then, we will focus on data stream
classification, including online learning. One
section will first briefly cover common
classifiers such as Naïve Bayes, kNN, and Decision
Trees. Another section will solely focus on Random
Forest.  The final section will summarize the two
preceding section and answer the question asked in
this introduction.

%How do we repatriate the data when it has been
%produce on a sensor on the far end of a sensor
%network? How do we do it efficiently?

%How to we mine those data? Which framework?
% vim: tw=50 ts=2
