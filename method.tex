\section{Method}
\subsection{Algorithms}
\paragraph{Hoeffding Tree~\cite{VFDT}}
\begin{itemize}
	\item Quick description of the idea.
	\item StreamDM-cpp implementation.
	\item OrpailleCC implementation.
	\begin{itemize}
		\item Reserved size with a given size.
		\item Binary tree.
		\item Focus on real numbers features.
		\item The number of split considered by features is given by the user.
		\item Split are determined by forming a boxes and spliting these boxes.
		\item Majority vote at the leaves.
		\item All floating point values are double and all counters are int.
	\end{itemize}
\end{itemize}
\paragraph{Micro Cluster Nearest Neighbor~\cite{mc-nn}}
\begin{itemize}
	\item Quick description of the idea.
	\item Specify the magic numbers (max cluster, error threshold, performance threshold).
	\item no specific details about the implementation.
\end{itemize}
\paragraph{Mondrian Forest~\cite{mondrian2014}}
\begin{itemize}
	\item Quick description of the idea.
	\item Size controlled by a buffer.
	\item Buffer shared by all trees.
	\item Specify the magic numbers (max lifetime, discount factor, base measure).
\end{itemize}
\paragraph{Naïve Bayes~\cite{naive_bayes}}
\begin{itemize}
	\item Very quick description of the idea.
	\item Same as before, the magic numbers (smoothing).
\end{itemize}
\paragraph{kNN~\cite{biased_reservoir_sampling}}
\begin{itemize}
	\item Use a biased reservoir sampling.
	\begin{itemize}
		\item Describe how the bias works.
	\end{itemize}
	\item Not implemented yet.
\end{itemize}

\subsection{Evaluation}
\paragraph{Datasets}
\begin{itemize}
	\item Banos~\cite{Banos_2014}.
	\begin{itemize}
		\item 50 Hz sampling.
		\item 117 data per sample.
		\item 33 activities.
		\item Sensors cover the body.
		\item Type of data (ideal or self)
		\item Number of subject.
	\end{itemize}
	\item Recofit~\cite{recofit}.
	\begin{itemize}
		\item Sampling rate?
		\item How many value per data point?
		\item How the sensors are placed?
		\item Number of subject?
	\end{itemize}
	\item Size of the window.
	\item Non-overlaping window.
	\item Which feature are extracted.
\end{itemize}

\paragraph{Subject Cross-Validation~\cite{subject_cross_validation}}
\begin{itemize}
	\item windows are grouped by subject.
	\item Each cross-validation iteration keeps one subject out.
	\item This subject is used as a test set.
	\item The average metric are computed for all iterations.
\end{itemize}

\paragraph{Prequential Error~\cite{issues_learning_from_stream}}
\begin{itemize}
	\item The idea is for each new data point, we test the model, then train the model with it.
	\item The loss function used.
\end{itemize}


\subsection{Metrics}
\begin{itemize}
	\item Time to process one element.
	\item Memory footprint
	\item Explain the concept drift recovery time.
	\item Energy consumption.
	\begin{itemize}
		\item Refer to rapl tools (\url{https://github.com/kentcz/rapl-tools.git}).
		\item Explain the methods to remove the activity of the computer.
		\item Intricate the measurement of algorithm and a sleep.
		\item Remove the average energy consumption of the sleep command to the average energy consumption to the algorithm.
		\item Randomize execution.
	\end{itemize}
\end{itemize}

