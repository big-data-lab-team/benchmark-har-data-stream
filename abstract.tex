\begin{center}
\section*{Abstract}
\begin{minipage}{0.9\textwidth}
%Background
The Internet of Things is a concept where smart
appliances collect data to improve services.
This concept applies to many scales, whether
it is a room or an entire city. Even though
these scales can cover wide areas,
the data produced are usually processed in a
centralized manner, which raises privacy and
energy efficiency issues.  These issues could
be addressed by mining data directly on
connected objects.  Even though numerous
problems have been elucidated in data stream
mining on small devices, the problem of
classification has not been fully covered.
This raises the question of designing
classification algorithms for data streams on
connected devices.
%Method
In this literature review, we first review the
main data stream mining algorithms, then we focus
on data stream classification.  We explain the
challenges related to data streams, and we study four
classification methods. For each of them, we
analyze their data stream variants and how they
address data stream challenges.
%Result
Results show that data stream classification is
already equipped with data stream tools from other
problems. These tools include sampling
methods, sliding windows, and online error
evaluation.  However, results also show that
classification algorithms are not always suited for data
stream challenges, especially the complex
learning models which often need too many
examples to learn, use too many resources and need
information not provided by a data stream
environment. 
%Conclusion
We conclude that the learning potential of
connected objects is limited by the lack of
complex models. Therefore, the type of
services that these appliances can provide is
also restrained. We note that CPU and memory
consumption should still be improve to port
complex models onto smart appliances.  Our
future research will study how bagging
methods, memory management algorithm, and
energy management system can be leveraged for
complex learning models.
\end{minipage}
\end{center}

% vim: tw=50 ts=2
