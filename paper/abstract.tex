\begin{center}
\section*{Abstract}
This paper aims to evaluate the feasibility of
running data stream classifiers on smart
devices. We start with a description of the
Internet of Things and how its centralized
pipeline limits the expansion of the concept.
We explain that leveraging data stream
algorithms on smart devices is a good
opportunity to tackle the limitation indued by
the centralized pipeline. Then we proposed to
focus on data stream classification in order
to evaluate classifiers adapted for this
purpose. In particular, we focus on their
learning curves and resource consumption
(runtime, memory, and power).
We retained eight classifiers that we wanted
to compare on two Human Activity datasets and
three additional synthetic datasets generated
by MOA. The results show that the runtime is
highly correlated to the classification
performances whereas the memory footprint
affects the memory bounded classifiers. The
power consumption shows no significant
variation regarding which classifier is run.
These results also show that the \naivebayes,
the \hoeffdingtree, and the \mondrianforest
has the best classification results. We
finally conclude by the need to improve the
\mondrianforest to decrease the impact of the
memory footprint impact or to reduce the
runtime.
\end{center}

% vim: tw=50 ts=2
