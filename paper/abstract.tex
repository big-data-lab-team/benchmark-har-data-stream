
\begin{abstract}
Decentralized data stream analyses have an important role to play in the
Internet of Things, through their potential impact on both users privacy
and energy consumption. This paper evaluates data stream classifiers from
the perspective of smart connected devices, focusing on the common use case
of human activity recognition. We measure both classification performance
and resource consumption (runtime, memory, and power) of eight usual stream
classification algorithms, implemented in a consistent library, and applied
to two real human activity datasets and three additional synthetic
datasets. Regarding classification performance, results show an overall
superiority of the \naivebayes, the \hoeffdingtree, and the \mondrianforest
classifiers over the \FNN and the Micro-Cluster Nearest Neighbor classifier
\TG{we lack a quantitative argument for that: X outperformed Y in a/b of
the datasets}. Regarding resource consumption, the runtime is highly
correlated to the classification performances whereas the memory footprint
affects the memory bounded classifiers \TG{be more explicit and name the
slowest / more memory intensive classifiers}. Power consumption shows no
significant variation among classifiers, although overall energy
consumption varies a lot due to differences in runtime. \TG{Conclusion on the feasibility of 
stream classification on connected objects for human activity recognition?} As future work,
decreasing the memory footprint of \mondrianforest seems to be a promising
way to improve classification performance while ensuring usability on connected objects.
\end{abstract}

% vim: tw=50 ts=2
