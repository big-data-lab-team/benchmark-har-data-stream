
\begin{abstract}
Decentralized data stream analyses have an important role to play in the
Internet of Things, through their potential impact on both users' privacy and
energy consumption. This paper evaluates data stream classifiers from the
perspective of smart connected devices, focusing on the common use case of
human activity recognition. We measure both classification performance and
resource consumption (runtime, memory, and power) of five usual stream
classification algorithms, implemented in a consistent library, and applied to
two real human activity datasets and three additional synthetic datasets.
Regarding classification performance, results show an overall superiority of
the \hoeffdingtree, the \mondrianforest, and the \naivebayes classifiers over
the \FNN and the \mcnn classifiers on 4 datasets out of
6 including the real datasets. The \hoeffdingtree
and to some extent \mcnn, are the only
classifiers that can recover from a concept
drift. Overall the F1-scores of the three
leading classifiers were quite low (lower than
0.7) on the real datasets.  On the contrary,
Regarding resource consumption, the
\hoeffdingtree and the \mondrianforest are the
most memory intensive, are the longest
runtime, and therefore are the highest energy
consummer since power does not vary among
classifiers. We conclude that Human Activity
Recognition on connected objects is set back
by two factors which could lead to interesting
research directions: a high memory consumption
and a low F1-scores overall.
\end{abstract}

% vim: tw=50 ts=2
