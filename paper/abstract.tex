
\begin{abstract}
% Data stream analyses have an important role to play in the
% Internet of Things, through the potential impact of decentralized architectures on both user privacy and
% energy consumption. 
This paper evaluates data stream classifiers from the
perspective of connected devices, focusing on the use case of
human activity recognition. We measure both classification performance and
resource consumption (runtime, memory, and power) of five usual stream
classification algorithms, implemented in a consistent library, and applied
to two real human activity datasets and to three synthetic datasets.
Regarding classification performance, results show an overall superiority
of the \hoeffdingtree, the \mondrianforest, and the \naivebayes classifiers
over the \FNN and the Micro Cluster Nearest Neighbor (\mcnn) classifiers on
4 datasets out of 6, including the real ones. In addition, the
\hoeffdingtree, and to some extent \mcnn, are the only classifiers that can
recover from a concept drift. Overall, the three leading classifiers still
perform substantially lower than an offline classifier on the real
datasets \TG{make sure this is discussed}. Regarding resource consumption, the \hoeffdingtree and the
\mondrianforest are the most memory intensive and have the longest runtime.
However, no difference in power consumption is found between classifiers. We
conclude that Human Activity Recognition on connected objects is set back
by two factors which could lead to interesting research directions: a high
memory consumption and low F1 scores overall.
\TG{harmonize capitalization of Human Activity Recognition throughout the document}
\end{abstract}


% vim: tw=80 ts=2
