
\begin{abstract}
Decentralized data stream analyses have an important role to play in the
Internet of Things, through their potential impact on both users' privacy and
energy consumption. This paper evaluates data stream classifiers from the
perspective of smart connected devices, focusing on the common use case of
human activity recognition. We measure both classification performance and
resource consumption (runtime, memory, and power) of eight usual stream
classification algorithms, implemented in a consistent library, and applied to
two real human activity datasets and three additional synthetic datasets.
Regarding classification performance, results show an overall superiority of
the \naivebayes, the \hoeffdingtree, and the \mondrianforest classifiers over
the \FNN and the Micro-Cluster earest Neighbor classifier on 4 datasets out of
6 including the real datasets. On the other datasets, we noted that \mcnn
showed good performances and tend to perform better than \naivebayes or the
\mondrianforest, especially when there was a concept drift. Despite their
overall superiority, we also noticed that the F1-scores of the three leading
classifiers were quite low on the real datasets.  Regarding resource
consumption, the \hoeffdingtree and the \mondrianforest have shown the highest
memory dependency. The \hoeffdingtree because it has the largest memory
footprint and the \mondrianforest because the more memory it has, the better
its F1-score is.  Additionally, the \mondrianforest and the \hoeffdingtree are
the slowest classifiers.  The fastest being the \mcnn and the \FNN.  Finally,
power consumption shows no significant variation among classifiers, although
overall energy consumption varies a lot due to differences in runtime.  We
conclude by stating that Human Activity Recognition on connected objects is set
back by two factors. A high need in memory to achieve the best F1-score, and
low F1-score despite higher memory.  As future work, we recommend decorrelating
the memory footprint and the F1-score of the \mondrianforest and the
\hoeffdingtree.  Additionally, we suggest improving the overall classification
performance of to make it usable.
\end{abstract}

% vim: tw=50 ts=2
