\documentclass[sensors,article,submit,moreauthors,pdftex]{Definitions/mdpi} 
\usepackage{graphicx}
\usepackage{hyperref}
\usepackage{url}
\usepackage{textcomp} %To avoid error with quote in biblio
\usepackage{xcolor}
\usepackage{ulem} % provides command 'uwave' used in command 'english'
\usepackage{tabularx}
\usepackage[font=small,skip=0pt]{caption}
\usepackage[font=small,skip=0pt]{subcaption}
\usepackage{lscape}
\usepackage{float}
\usepackage{tikz}
\usepackage{dblfloatfix} %To force figure* at the bottom of page without shifting everything
\usepackage{grffile} %Used to allow filename with many dots 
\usepackage{float} %Used to force figure placement
\usepackage{xspace} %Used in \banosdataset
\usepackage{xcolor}
% Fix link colors
\hypersetup{
    colorlinks = true,
    linkcolor=red,
    citecolor=red,
    urlcolor=blue,
    linktocpage % so that page numbers are clickable in toc
}
\newcommand{\rand}[1]{
\pgfmathparse{random(#1)}\pgfmathresult }

\newcommand{\TG}[1]{\noindent{\color{blue}[\textsc{From Tristan:} #1]}}
\newcommand{\english}[1]{\uwave{#1}} % Indicates that the sentence has syntax errors or uses weird words
\newcommand{\writing}[1]{\color{red}#1\color{black}} % Indicates that the sentence is unlcear, heavy, or has too many words
\newcommand{\MK}[1]{\noindent{\color{red}[\textsc{M.Khannouz:} #1]}}
\newcommand{\banosdataset}[0]{Banos \textit{et al}\xspace}
\newcommand{\recofitdataset}[0]{Recofit\xspace}
\newcommand{\hoeffdingtree}[0]{Hoeffding Tree\xspace}
\newcommand{\mondrianforest}[0]{Mondrian forest\xspace}
\newcommand{\mondriantree}[0]{Mondrian tree\xspace}
\newcommand{\mondriantrees}[0]{Mondrian trees\xspace}
\newcommand{\naivebayes}[0]{Naïve Bayes\xspace}
\newcommand{\FNN}[0]{Feedforward Neural Network\xspace}
\newcommand{\mcnn}[0]{MCNN\xspace}
\newcommand{\mcnns}[0]{MCNNs\xspace}
\newcommand{\har}[0]{HAR\xspace}
\newcommand{\knn}[0]{\textit{k}-NN\xspace}
\newcommand{\streamdmcpp}[0]{StreamDM-C++\xspace}
\definecolor{revcolor}{RGB}{255,225,255}
\newcommand{\revision}[1]{\colorbox{revcolor} {\parbox{\textwidth}{#1}}}


\Title{{A benchmark of data stream classification
for human activity recognition
on connected objects}}

\firstpage{1} 
\makeatletter 
\setcounter{page}{\@firstpage} 
\makeatother
\pubvolume{xx}
\issuenum{1}
\articlenumber{5}
\pubyear{2020}
\copyrightyear{2020}
%\externaleditor{Academic Editor: name}
\history{Received: date; Accepted: date; Published: date}

% Author Orchid ID: enter ID or remove command
\newcommand{\orcidauthorA}{0000-0003-2129-5517} % Add \orcidA{} behind the author's name
\newcommand{\orcidauthorB}{0000-0003-2620-5883} % Add \orcidB{} behind the author's name

% Authors, for the paper (add full first names)
\Author{Martin Khannouz$^1$\orcidA{}, Tristan Glatard$^1$\orcidB{}}


% Authors, for metadata in PDF
\AuthorNames{Martin Khannouz and Tristan Glatard}

% Affiliations / Addresses (Add [1] after \address if there is only one affiliation.)
\address{$^{1}$ \quad Department of Computer Science and Software Engineering, Concordia University}

% Contact information of the corresponding author
\corres{Correspondence: martin.khannouz@gmail.com, tristan.glatard@concordia.ca}

% Current address and/or shared authorship
%\firstnote{Current address: Affiliation 3} 
%\secondnote{These authors contributed equally to this work.}
% The commands \thirdnote{} till \eighthnote{} are available for further notes

%\simplesumm{} % Simple summary

\abstract{
This paper evaluates data stream classifiers from
the perspective of connected devices, focusing on
the use case of Human Activity Recognition. We
measure both classification performance and
resource consumption (runtime, memory, and power)
of five usual stream classification algorithms,
implemented in a consistent library, and applied
to two real human activity datasets and to three
synthetic datasets.  Regarding classification
performance, results show an overall superiority
of the Hoeffding Tree, the Mondrian forest, and
the Naïve Bayes classifiers over the Feedforward
Neural Network and the Micro Cluster Nearest
Neighbor classifiers on 4 datasets out of 6,
including the real ones. In addition, the
Hoeffding Tree, and to some extent the Micro
Cluster Nearest Neighbor, are the only classifiers
that can recover from a concept drift. Overall,
the three leading classifiers still perform
substantially lower than an offline classifier on
the real datasets. Regarding resource consumption,
the Hoeffding Tree and the Mondrian forest are the
most memory intensive and have the longest
runtime, however, no difference in power
consumption is found between classifiers. We
conclude that stream learning for Human Activity
Recognition on connected objects is challenged by
two factors which could lead to interesting future
work: a high memory consumption and low F1 scores
overall.
}

% Keywords
\keyword{
Application Platform; Data Management and
Analytics; Smart Environment; Data
Streams; Classification; Power; Memory Footprint;
Benchmark; Human Activity
Recognition; MCNN; Mondrian; Hoeffding Tree;}
\begin{document}
\maketitle

\begin{abstract}
Decentralized data stream analyses have an important role to play in the
Internet of Things, through their potential impact on both users' privacy and
energy consumption. This paper evaluates data stream classifiers from the
perspective of smart connected devices, focusing on the common use case of
human activity recognition. We measure both classification performance and
resource consumption (runtime, memory, and power) of five usual stream
classification algorithms, implemented in a consistent library, and applied to
two real human activity datasets and three additional synthetic datasets.
Regarding classification performance, results show an overall superiority of
the \hoeffdingtree, the \mondrianforest, and the \naivebayes classifiers over
the \FNN and the \mcnn classifiers on 4 datasets out of
6 including the real datasets. The \hoeffdingtree
and to some extent \mcnn, are the only
classifiers that can recover from a concept
drift. Overall the F1-scores of the three
leading classifiers were quite low (lower than
0.7) on the real datasets.  On the contrary,
Regarding resource consumption, the
\hoeffdingtree and the \mondrianforest are the
most memory intensive, are the longest
runtime, and therefore are the highest energy
consummer since power does not vary among
classifiers. We conclude that Human Activity
Recognition on connected objects is set back
by two factors which could lead to interesting
research directions: a high memory consumption
and a low F1-scores overall.
\end{abstract}

% vim: tw=50 ts=2

\section{Introduction}
\label{sec:introduction}

Internet of Things applications may adopt a
centralized model, where connected objects transfer data to servers with 
adequate computing capabilities, or a decentralized model, where data is analyzed directly on the connected
objects or on nearby devices. While the decentralized model limits network transmission,
increases battery life~\cite{sensor-network-survey, sensor-energy-model},
and reduces data privacy risks, it also raises important processing challenges
due to the modest computing capacity of connected objects. Indeed, it is not uncommon for wearable devices and other smart objects
to include a processing memory of less than 100~KB, little to no storage
memory, a slow CPU, and no operating system. With multiple sensors
producing data at high frequencies, typically 50~Hz to 800~Hz, processing speed and
memory consumption become critical properties of data analyses. 

Data stream processing algorithms are precisely designed to analyze
virtually infinite sequences of data elements with reduced amounts of
working memory. Several classes of stream processing algorithms were
developed in the past decades, such as filtering, counting, or sampling
algorithms~\cite{kejariwal2015}. \TG{Maybe mention a few key properties of streaming algorithms} Our study focuses on supervised
classification, a key component of contemporary data models.

We evaluate supervised data stream classifiers from
the point of view of connected objects, with a particular focus on human
activity recognition. The main motivating use case is that of wearable
sensors measuring 3D acceleration and orientation at different locations on
the human body, from which activities such as gym exercises have to be
predicted. A previously untrained supervised classifier is deployed directly on the wearables or on
a nearby object, perhaps a watch, and aggregates the data, learns a data model, predicts the current
activity, and episodically receives true labels from the human subject. Our
main question is to determine whether on-chip classification is feasible in
this context. 

We evaluate existing classifiers from the complementary angles of (1)
classification performance, including in the presence of concept drift, and
(2) resource consumption, including memory usage and classification time
per element (latency). We consider six datasets in our benchmark, including
the two most popular open datasets used for human activity recognition, and
four simulated datasets. 


%NOTE: Keep the next three lines :D.
%\cite{sensor-energy-consumption} (conclusion, second paragraph) communication uses more energy.
%\cite{leach} and \cite{sensor-energy-model}(page 3, first column, check equations and values)
%\cite{sensor-network-survey} : "Since the sensor nodes are often inaccessible, the lifetime of a sensor network depends on the lifetime of the power resources of the nodes"

% offline works
Several studies evaluated classifiers for human activity recognition in an
offline (non data stream) setting. In particular, the work
in~\cite{Janidarmian_2017} compared 293 classifiers using various sensor
placements and window sizes, concluding on the superiority of k nearest
neighbors (kNN) and pointing out a trade-off between runtime and
classification performance. Resource consumption, including memory and runtime, was also studied 
for offline classifiers, such as in~\cite{memory_consumption_machine_learning}
for the particular case of the R programming language.

% data stream classifiers
Data stream classifiers were also compared, in particular
in~\TG{Martin, add references and a brief summary of results.}
However, these studies remained limited to ... 

Regarding connected objects, the work in~\cite{omid_2019} presents a
wearable system capable of running pre-trained classifiers \TG{Martin,
are they really pre-trained?} with high classification accuracy. It shows
the superiority of the proposed Feedforward Neural Network~(FNN) over KNN
~\TG{Check the spelling of kNN vs KNN and harmonize accordingly}. \TG{any other relevant study on connected objects?}
%  It shows that a trained neural
% network achieves high accuracy and performs better
% than KNN. The dataset was acquired with the
% Neblina, a wearable sensor placed on the right
% forearm. The data from the Neblina was merged to
% send to the computer a stream of 9-axis. Only
% then, the stream was processed to extract features
% and feed the classifier.

Compared to these previous works, the contributions of our paper are the following:
\begin{itemize}
    \item We compare the most popular data stream classifiers on the specific case of human activity recognition;
    \item We provide quantitative measurements of memory and power consumption, as well as runtime;
    \item We implement data stream classifiers in a consistent software library meant for deployment on embedded systems.
\end{itemize} 
The subsequent sections present the materials, methods, and results of our benchmark.

%Étendre related work, regarder les papiers qui ont cité ces papier
%En particulier la référence Janidarmian_2017, pareil pour memory_consumption_machine_learning

% vim: tw=50 ts=2

\section{Method}
We evaluated 5 classifiers implemented in either
StreamDM-Cpp~\cite{StreamDM-CPP} or
OrpailleCC~\cite{OrpailleCC}.  StreamDM-Cpp is a
C++ implementation of StreamDM~\cite{StreamDM}, a
software to mine big data streams using
\href{https://spark.apache.org/streaming/}{Apache
Spark Streaming}. StreamDM-Cpp is usually faster
than StreamDM in single-core environments, due to the
overhead induced by Spark.

OrpailleCC is a collection of data stream
algorithms developed for embedded devices. The key
functions, such as random number generation or
memory allocation, are parametrizable through
class templates and can thus be customized on a
given execution platform.  OrpailleCC is not
limited to classification algorithms, it
implements other data stream algorithms such as
the Cuckoo filter~\cite{cuckoo} or a
multi-dimensional extension of the Lightweight
Temporal Compression~\cite{multi-ltc}. We extended
it with a few classifiers for the purpose of this
benchmark.

This benchmark includes five popular classification
algorithms.  The
\mondrianforest~\cite{mondrian2014} builds
decision trees without immediate need for labels
which is useful in situations where labels are
delayed~\cite{stream_learning_review}.  The
Micro-Cluster Nearest
Neighbors~\cite{mc-nn} is a compressed version of the k-nearest
neighbor~(KNN) that was shown to be among the most accurate classifiers for human activity
recognition from wearable sensors~\cite{Janidarmian_2017}. The \naivebayes~\cite{naive_bayes}
classifier builds a table of feature
occurrence to estimate class
likelihoods.
The \hoeffdingtree~\cite{VFDT} builds a
decision tree using the Hoeffding Bound to
estimate when the best split is found. 
Finally, Neural Network classifiers have
become popular by reaching or even exceeding human performance in many fields such as image
recognition or game playing. We used a
\FNN with one hidden layer, as described in~\cite{omid_2019} for the recognition 
of fitness activities on a low-power platform.

The remainder of this section details the datasets, classifiers,
evaluation metrics and parameters used in our benchmark.

\subsection{Datasets}
\label{sec:method-dataset}
\subsubsection{\banosdataset}
%50 Hz sampling.
%117 data per sample.
%33 activities.
%Sensors cover the body.
%Type of data (ideal or self)
%17 subject.
The \banosdataset dataset~\cite{Banos_2014} is a
human activity dataset with 17 participants
and 9 sensors per
participant\footnote{\banosdataset dataset
available
\href{https://archive.ics.uci.edu/ml/datasets/REALDISP+Activity+Recognition+Dataset\#:\~:text=The\%20REALDISP\%20(REAListic\%20sensor\%20DISPlacement,\%2Dplacement\%20and\%20induced\%2Ddisplacement.}{here}.}. Each sensor samples a 3D
acceleration, gyroscope, and magnetic field, as
well as the orientation in a quaternion format,
producing a total of 13 values.  Sensors are
sampled at 50~Hz, and each sample is associated
with one of 33 activities. In addition to the 33
activities, an extra activity labelled 0 indicates
no specific activity.

We pre-processed the \banosdataset dataset using
non-overlapping windows of one second (50
samples), and using only the 6 acceleration axes
of the right forearm sensor. We computed the average and the standard deviation over the
window as features for each axis. We assigned the most
frequent label to the window.  The resulting data
points were shuffled uniformly.

In addition, we constructed another dataset from \banosdataset, in which we
simulated a concept drift by shifting the activity labels in the
second half of the data stream. This is useful to
observe any behavioral change induced by the
concept drift such as an increase in power
consumption.

\subsubsection{\recofitdataset}
The \recofitdataset dataset~\cite{recofit} is a
human activity dataset containing 94
participants\footnote{\recofitdataset dataset
available
\href{https://msropendata.com/datasets/799c1167-2c8f-44c4-929c-227bf04e2b9a}{here}.}. Similarly to the \banosdataset
dataset, the activity labelled 0 indicates no
specific activity.
Since many of these activities are similar, we
merged some of them together based on the table
in~\cite{behzad2019}. 

We pre-processed the dataset similarly to the
\banosdataset one, using non-overlapping windows of
one second, and only using the 6 acceleration axes
of the 6 axis data. From these 6 axes, we used the average and the standard deviation
over the window as features. We assigned the most
frequent label to the window.

\subsubsection{MOA dataset}
Massive Online Analysis~\cite{moa} (MOA) is a Java framework to compare
data stream classifiers. In addition to classification algorithms, MOA provides many
tools to read and generate datasets.
We generated three synthetic datasets\footnote{MOA commands available \href{https://github.com/azazel7/paper-benchmark/blob/e0c9a94d0d17490f7ab14293dec20b8322a6447c/Makefile\#L90}{here}.}:
a hyperplane, a RandomRBF, and a RandomTree
dataset. We generated 200,000 data points
 for each of these synthetic datasets.
The hyperplane and the RandomRBF both have three features and two classes, however, the RandomRBF has a slight imbalance toward one class.
The RandomTree dataset is the hardest of the three, with six attributes and
ten classes. Since the data points are generated with a tree structure, we
expect the decision trees to show better performances than the other
classifiers.

\subsection{Algorithms and Implementation}
In this section, we describe the algorithms used in the benchmark, their
hyperparameters, and relevant implementation details. 

\subsubsection{\mondrianforest~\cite{mondrian2014}}
Each tree in a \mondrianforest recursively splits
the feature space, similar to a regular decision tree.
However, the feature used in the split and the
value of the split are picked randomly. The
probability to select a feature is proportional to
its normalized range. The value for the split is
uniformly selected in the range of the feature.
During prediction, a node combines its observed
label count with its parent prediction.

In OrpailleCC, the size allocated for the forest
is set beforehand and it is shared by all
the trees.  Therefore, the memory footprint of the
classifier is constant.

Mondrian trees can be tuned using three
parameters: the base count, the discount factor,
and the budget. The base count is used to
initialize the prediction for the root. The
discount factor influences the nodes on how much
they should use their parent prediction. A
discount factor closer to one makes the prediction
of a node closer to the prediction of its parent.
Finally, the budget controls the tree depth.

Table~\ref{table:hyperparameter-mondrian} shows
the hyperparameters used for each number of trees.
\begin{figure}
	\begin{center}
		\begin{tabular}{|| r | c | c | c ||} 
			\hline
			Number of trees &  Base count & Discount & Budget \\ [0.5ex] 
			\hline\hline
			1 & 0.0 & 1.0 & 1.0 \\
			5 & 0.0 & 1.0 & 0.4 \\
			10 & 0.0 & 1.0 & 0.4 \\
			50 & 0.0 & 1.0 & 0.2 \\
			\hline
		\end{tabular}
	\end{center}
	\caption{Hyperparameters used for \mondrianforest.}
		\label{table:hyperparameter-mondrian}
\end{figure}
\subsubsection{Micro Cluster Nearest Neighbor~\cite{mc-nn}}
The Micro Cluster Nearest Neighbor~(\mcnn) is a
variant of k-nearest neighbors where data points
are aggregated into clusters to reduce storage
requirements.  During training, the algorithm
merges a new data point to the closest cluster
that shares the same label. If the closest cluster
does not share the same label as the data point,
this closest cluster and the closest cluster with
the same label are assigned an error. When a
cluster receives too many errors, it is split.
During classification, \mcnn returns the label of
the closest cluster.  Regularly, the algorithm
also assigns a participation score to each cluster
and when this score gets below a threshold, the
cluster is removed. Given that the maximum number
of clusters is fixed, this mechanism makes
space for new clusters, and possibly
helps adjust to concept drifts.  
The space and time complexity of \mcnn are
constant since the maximum number of clusters does
not change.

We implemented two versions of \mcnn in
OrpailleCC, which differ in the way they remove
clusters during training. The first version (\mcnn
Origin) is similar to the mechanism described
in~\cite{mc-nn}, based on participation scores.
The second version (\mcnn OrpailleCC)
removes the cluster with the lowest participation
only when space is needed.  A cluster slot is
needed when an existing cluster is split and there
are no more slot available because the number of
active clusters already reached the maximum defined
by the user.

\mcnn OrpailleCC has one
parameter, the error threshold after which a
cluster is split.  \mcnn Origin has two
parameters: the error threshold and the
participation threshold. The participation
threshold is the limit below which a cluster is
removed.

Table~\ref{table:hyperparameter-mcnn} shows
the hyperparameters used for each number of clusters \TG{where is the participation threshold?}.
\begin{figure}
		\begin{center}
			\begin{tabular}{|| r | c ||} 
				\hline
				Number of clusters &  error threshold \\ [0.5ex] 
				\hline\hline
				10 & 2 \\
				20 & 10 \\
				33 & 16 \\
				40 & 8 \\
				50 & 2 \\
				\hline
			\end{tabular}
		\end{center}
		\caption{Hyperparameters used for \mcnn.}
		\label{table:hyperparameter-mcnn}
\end{figure}

\subsubsection{\naivebayes~\cite{naive_bayes}}
The \naivebayes algorithm keeps a table of
counters for each feature value and each label.
During prediction, the algorithm assigns a
score for each label depending on how the data
point to predict compares to the values observed
during the training phase.

The implementation from StreamDM was used in this
benchmark. It uses a Gaussian
fit for numerical attributes.

In a \naivebayes classifier, the smoothing parameter is the
default counter given to an attribute value that
has not been seen. It is meant to avoid scores of zeros.
We do not use \TG{could you check that all the verbs that describe the experiment are present tense? 
A few of them are past.} any smoothing since our datasets
only contain numerical
attributes. Therefore, as long as there is one data
point to train with, there will be a Gaussian
fit for each attribute.

\subsubsection{\hoeffdingtree~\cite{VFDT}}
%\begin{itemize}
	%\item Reserved size with a given size.
	%\item Binary tree.
	%\item Focus on real numbers features.
	%\item The number of split considered by features is given by the user.
	%\item Split are determined by forming a boxes and spliting these boxes.
	%\item Majority vote at the leaves.
	%\item All floating point values are double and all counters are int.
%\end{itemize}
Similar to a decision tree, the \hoeffdingtree
recursively splits the feature space to maximize a metric, often the
information gain or the Gini
index. However,  to estimate when a leaf should be
split, the \hoeffdingtree relies on the
Hoeffding bound, a measure of the score deviation
of the splits instead of using the entire
dataset \TG{unclear: what is using the entire dataset? decision trees only focus on the data in the leaf to split}. During classification, a data point
is sorted to a leaf, and a label is predicted by
aggregating the labels of the training data points
in that leaf, usually through majority voting or
\naivebayes classification.  We used this
classifier as implemented in StreamDM-Cpp.  The
\hoeffdingtree is common in data stream
classification, however, it suffers from one main
disadvantage: once a split is decided, it cannot
be re-considered which makes this algorithm weak
to concept drifts \TG{This is a weird comment given that Hoeffing Tree is
the only classifier that recovers well from the drift in the results. Also, there's no mentioning of 
concept drifts in the other classifiers, maybe just add a paragraph to comment in general about their 
robustness to concept drift.}.

The \hoeffdingtree has multiple \TG{say three explicitly} parameters: the
confidence level, the grace period, and the leaf
learner. The confidence level is probability that
the Hoeffding bound makes a wrong estimation of
the deviation. The grace period is the number of
processed data points before a leaf is evaluated for split.
 The leaf learner is the method used in the
leaf to predict the label.  In this study we used
a confidence level of $0.01$ with a grace period
of 10 data points and \naivebayes as leaf
learner.

\subsubsection{\FNN}
%We need a citation there
A neural network is a combination of artificial
neurons, also known as perceptrons, that all have input weights and an
activation function. To predict a class label, the
perceptron applies the activation function to the weighted sum
of its input values. The output
value of the perceptron is the result of this
activation function. This prediction phase is also
called feed-forward. To train the neural network,
feed-forward is applied first, then the error between the
prediction and the expected result is used in the
backpropagation process to adjust the weights of
the input values.  A neural network combines
multiple perceptrons by connecting perceptron outputs
to inputs of other perceptrons.  In
this benchmark, we used a fully-connected \FNN, 
that is, a network where perceptrons are organized in
layers and all output
values from perceptrons of layer $n-1$ serve as
input values for perceptrons of layer $n$. 
We used a 3-layer network with 120 inputs, 30
perceptrons in the hidden layer, and 33 output
perceptrons.

In this study, we used histogram features
from~\cite{omid_2019} instead of the ones
presented in Section~\ref{sec:method-dataset}
because the network performed
poorly with these features. The histogram features
produce 20 bins per axis.

This neural network can be tuned by changing the
number of layers and the size of each layer.
Additionally, the activation function and the
learning ratio can be changed. The learning ratio
indicates by how much the weights should change
during backpropagation.

\subsubsection{Hyperparameters Tuning}
For each classifier, hyperparameters were tuned using the first
subject from the \banosdataset dataset.  The data from
this subject was pre-processed as the rest of
the \banosdataset dataset (window size of one second,
average and standard deviation on the three
acceleration axis of the right forearm sensor,
$\ldots$). We tested multiple values for the
parameters \TG{grid search?}.

The
classifiers start the prequential phase with no
knowledge from the first subject.
We made an exception for the \FNN because we noticed that it performed poorly with random weights and it
needed many epochs to achieve better performances
than a random  classifier. Therefore, we decided to
pre-train the \FNN and re-use
the weights as a starting point for the
prequential phase.

For other classifiers, only the 
hyperparameters were taken from the tuning phase.
We selected the hyperparameters that 
maximized the F1-score on the first subject.


\subsection{Evaluation}
We computed four metrics: the F1-score, the memory
footprint, the runtime, and the power usage.
The F1-score and the memory
footprint were computed periodically during the
execution of a classifier. The
power consumption and the runtime were collected
at the end of each execution.

%Time to process one element.
%Explain the concept drift recovery time.
\paragraph{Classification Performance}
We monitored the true positives, false positives,
true negatives, and false negatives using the
prequential evaluation, meaning that with each
new data point the model was first tested and then trained.
From these counts, we computed the F1-score every
50 elements. We did not apply any fading factor
to attenuate errors throughout the stream.
We computed the F1-score in a one-versus-all
fashion for each class, averaged across all
classes
(macro-average, code available \href{https://github.com/azazel7/paper-benchmark/blob/9adb1039c5a65a00a66d554f0e870d14d3fff7cb/main.cpp\#L82}{here}). 
When a class had not been encountered yet, its F1-score was ignored. We used
the F1-score rather than the accuracy because the real data sets are
imbalanced.

\paragraph{Memory}
We measured the memory footprint by reading file
\texttt{/proc/self/statm} every 50 data points.

\paragraph{Runtime}
The runtime of a classifier is the time needed for
the classifier to process the dataset. It is
measured at the end of an execution using the
\textit{perf}
command \TG{maybe naive but how do you measure runtime only at the end of
the execution? Maybe reword that.} \TG{There is still an issue with verb
tenses here: ``It is measured'' in this paragraph, ``We measured'' in the
previous one.}. \TG{Add link to the tool}

%\MK{Explain the runtime somewhere and how it is
%computed.}

\paragraph{Power} We measured the average power
consumed by classification algorithms using the
\textit{perf} command. We measured the power used by each
classifier multiple times in a minimal
environment. To provide a baseline power usage, we
also added an empty classifier that always
predicted class 0.

\subsection{Experimental Conditions}
We automated our experiments with a Python script that defines
classifiers and their parameters, randomizes all
the repetitions, and plots the
resulting data. The datasets and output results are stored in memory
through a memfs filesystem mounted on \texttt{/tmp}, to reduce the impact of I/O time.
We averaged metrics accross repetitions (same classifier, same parameters, and
same dataset).

The experiment was done  with a single core of a
cluster node with two Intel(R) Xeon(R)
Gold 6130 CPUs and a main memory of 250G.

% vim: tw=50 ts=2

\begin{figure*}
	\begin{subfigure}[t]{.49\linewidth}
		\includegraphics[width=\linewidth]{figures/results/banos_6_f1.png}
		\caption{\banosdataset}
		\label{fig:f1-banos}
	\end{subfigure}
	\hfill
	\begin{subfigure}[t]{.49\linewidth}
		\includegraphics[width=\linewidth]{figures/results/recofit_6_f1.png}
		\caption{\recofitdataset}
		\label{fig:f1-recofit}
	\end{subfigure}\\
	\begin{subfigure}[t]{.49\linewidth}
		\includegraphics[width=\linewidth]{figures/results/dataset_1_f1.png}
		\caption{Hyperplane (MOA)}
		\label{fig:f1-dataset_1}
	\end{subfigure}
	\hfill
	\begin{subfigure}[t]{.49\linewidth}
		\includegraphics[width=\linewidth]{figures/results/dataset_2_f1.png}
		\caption{RandomRBF (MOA)}
		\label{fig:f1-dataset_2}
	\end{subfigure}\\
	\begin{subfigure}[t]{.49\linewidth}
		\includegraphics[width=\linewidth]{figures/results/dataset_3_f1.png}
		\caption{RandomTree (MOA)}
		\label{fig:f1-dataset_3}
	\end{subfigure}
	\hfill
	\begin{subfigure}[t]{.49\linewidth}
		\includegraphics[width=\linewidth]{figures/results/drift_3_f1.png}
		\caption{\banosdataset (with Drift)}
		\label{fig:f1-drift}
	\end{subfigure}
	\caption{F1-scores for the six datasets. 
	\TG{Hoeffding Tree is a bit difficult to see because it doesn't have a marker and the colour is a bit pale}
	\TG{You should explain what the MCNN and Mondrian params are, I think just adding ``trees'' (ex: ``10 trees'' and ``clusters'' (ex: ``10 clusters'' in the legend would be enough))}}
	\label{fig:f1}
\end{figure*}

\section{Results}

This section presents the results of our benchmark and the corresponding
hyperparameter tunning experiments.

\subsection{F1-score}
Figure~\ref{fig:f1} compares the F1-scores obtained by all classifiers on the
six datasets.  \naivebayes and the \hoeffdingtree outperforme the other
classifiers on the two real datasets (\banosdataset and \recofitdataset).  On
the other hand, the \mondrianforest with 5 or 10 trees achieves the best
performances on 2 synthetic datasets.  Also, the \mondrianforest is the third
classifier on the two real datasets.  Finally, the \hoeffdingtree outperform
all other classifiers on the RandomTree dataset, where it is followed by the
\mondrianforest.

\TG{shall we call them ``Mondrian Forest'' consistently throughout the paper? 
Currently they are also called ``Mondrian Trees'' or just ``Mondrian''}

F1-score values vary greatly across the datasets.  While the highest
observed F1-score is above 0.95 on the Hyperplane and RandomRBF datasets,
it barely reaches 0.65 for the \banosdataset dataset, and it remains under
0.4 on the \recofitdataset and RandomTree datasets. This trend is
consistent for all classifiers.

The StreamDM \hoeffdingtree algorithm achieves better performance than the
\naivebayes except for the \banosdataset dataset.  Both to them start close
together because the \hoeffdingtree uses a \naivebayes in its leaves.  However,
they start diverging because the \hoeffdingtree improves by reshaping its tree
structure.  This is caused by a sufficient amount of element and the difference
is more noticable when a concept drift occurs.

On all datasets, \mcnn OrpailleCC achieves better performances than \mcnn
Original. Additionally, on the datasets RandomRBF and RandomTree, the lowest
\mcnn OrpailleCC performs better than the highest \mcnn Origin \TG{not sure I
understand this sentence. Maybe add a brief explanation instead, ``presumably
due to \ldots''}.

On the real datasets (\banosdataset and \recofitdataset), the \mcnn OrpailleCC
classifier appears to be learning faster than the \mondrianforest, although
\mondrianforest catch up after a few thousand elements. 

Surprisingly, a \mondrianforest with 50 trees performs worse than with 5 or 10
trees. This is due to the fact that our \mondrianforest implementation forces a
fixed memory footprint, which limits tree growth when the allocated memory is
full. Because 50 trees fill the memory faster than 10 or 5 trees, the
classifier adaptation is blocked faster, when the trees have not learned enough
from the data.

Figure~\ref{fig:f1-banos} and Figure~\ref{fig:f1-recofit} show the F1-score
difference between. The \mondrianforest in the dark blue plot uses twice as
much memory as the on cyan plot. 

The \hoeffdingtree appears to be the most robust to concept drifts
(Figure~\ref{fig:f1-drift}), while the \mondrianforest and \naivebayes
classifier are the most impacted. \mcnn classifiers are marginally impacted.
The low resilience of \mondrianforest to concept drifts can be attributed to
two factors. First, \mondrianforest cannot update existing nodes, only add new ones.
Second, when the memory limit is reached, \mondrianforest are not able to grow
or reshape their structure anymore.

Figure~\ref{fig:f1-banos} shows that the F1-score of the Multi-Layer Perceptron
is around 0.3. Therefore, it remains better than \mcnn Origin but it is quite
low compare to other classifiers with the same dataset.

Finally, we notice that the StreamDM and OrpailleCC implementations of \naivebayes remain close on the two real datasets: \banosdataset and \recofitdataset.
This suggests that the two implementations are not similar, but SreamDM
implements additional mechanisms \TG{this is too vague}.

\TG{Paragraphs should be added to comment on :
\begin{itemize}
	\item neural networks
\end{itemize}}


\begin{figure*}
	\begin{subfigure}[t]{.49\linewidth}
		\includegraphics[width=\linewidth]{figures/results/banos_3_watt.png}
		\caption{\banosdataset}
		\label{fig:power-banos}
	\end{subfigure}
	\hfill
	\begin{subfigure}[t]{.49\linewidth}
		\includegraphics[width=\linewidth]{figures/results/recofit_3_watt.png}
		\caption{\recofitdataset}
		\label{fig:power-recofit}
	\end{subfigure}\\
	\begin{subfigure}[t]{.49\linewidth}
		\includegraphics[width=\linewidth]{figures/results/drift_3_watt.png}
		\caption{\banosdataset with drift.}
		\label{fig:power-drift}
	\end{subfigure}
	\hfill
	\begin{subfigure}[t]{.49\linewidth}
		\includegraphics[width=\linewidth]{figures/results/dataset_3_watt.png}
		\caption{RandomTree}
		\label{fig:power-dataset_3}
	\end{subfigure}
	\caption{Power usage for four datasets.}
	\label{fig:power}
\end{figure*}
\subsection{Power}
\label{sec:result-power}
Figure~\ref{fig:power} shows the power usage of each classifier on four
datasets \TG{mention why the 6 datasets aren't reported}. Since all classifiers
exhibit comparable power consumptions, close to 102~W, we decided to show only
four of them.


\subsection{Runtime}
Figure~\ref{fig:runtime} shows the runtime of classifiers for the two real
datasets \TG{you haven't explained in the methods how runtime was
measured}. \mondrianforest are the slowest classifier, in particular for
50 trees. The second slowest classifier is the \hoeffdingtree, which
compares with the 1-tree \mondrianforest. The \hoeffdingtree is followed by
the two \naivebayes implementations, which is not surprising since \naivebayes classifiers are used in the leaves of the \hoeffdingtree. The \mcnn
classifiers are the fastest ones, with a runtime very close to the empty
classifier. There is still a slight difference between the \mcnns. The more
cluster we use, the slower it gets \TG{at this scale it is barely
noticeable, I would ommit this comment.}.

We observe that \naivebayes from StreamDM is slightly faster than the one
from OrpailleCC \TG{I think they are quite comparable in fact. You should
explain that this makes the comparison between hoeffding tree and mondrian
fair even though they are implemented differently.}.

\begin{figure*}
	\begin{subfigure}[t]{.49\linewidth}
		\includegraphics[width=\linewidth]{figures/results/banos_6_runtime.png}
		\caption{\banosdataset}
		\label{fig:runtime-banos}
	\end{subfigure}
	\hfill
	\begin{subfigure}[t]{.49\linewidth}
		\includegraphics[width=\linewidth]{figures/results/recofit_6_runtime.png}
		\caption{\recofitdataset}
		\label{fig:runtime-recofit}
	\end{subfigure}
	\caption{Runtime with the two real datasets. \TG{y-axis should be ``Dataset processing time (seconds)''}}
	\label{fig:runtime}
\end{figure*}

\subsection{Memory}
\label{sec:result-memory}
Figure~\ref{fig:memory} shows the evolution of the memory footprint for the
\banosdataset dataset.  Memory footprint is similar across all datasets, due to the fact that most algorithms
follow a bounded memory policy or have a constant space complexity. The only
exception is the \hoeffdingtree \TG{check that spelling and capitalization
of \hoeffdingtree is consistent throughout document} that constantly
selects new splits depending on 
new data points. The \mondrianforest would have had the same behavior if it
would allocate memory as element goes by. However, the \mondrianforest implementation
of OrpailleCC allocates memory when the classifier is instantiated and does not
use more \TG{Last two sentences should be improved, this can be written in a more compact way.}.

\begin{figure}
	\includegraphics[width=\linewidth]{figures/results/banos_3_memory.png}
	\caption{Memory used by the algorithms on the Banos dataset. \TG{increase font size in all graphs, to make it only slightly smaller than font size in caption}}
	\label{fig:memory}
\end{figure}


\subsection{Micro-Cluster Nearest Neighbor Hyperparameters}

Figure~\ref{fig:mcnn-tuning-error} shows the impact of the error threshold
in the \mcnn classifiers. The higher the number of clusters, the better are
the performance \TG{ok but the number of clusters isn't a hyper parameter,
this should be moved to the description of F1 score results}. The error
threshold of \mcnn has little impact on the classification performance. For
20 and 40 clusters, the best-performing threshold is either 2 or 4, meaning
that a cluster may do 2 or 4 errors before being split. For 10 clusters,
all error thresholds perform equally.

\begin{figure}
	 \begin{subfigure}[b]{0.49\textwidth}
		 \centering
		 \includegraphics[width=\linewidth]{figures/Banos_S1_shuf_MCNN_40_error_check.png}
		 \caption{40 clusters}
	 \end{subfigure}
	 \begin{subfigure}[b]{0.49\textwidth}
		 \centering
		 \includegraphics[width=\linewidth]{figures/Banos_S1_shuf_MCNN_20_error_check.png}
		 \caption{20 clusters}
	 \end{subfigure}
	 \begin{subfigure}[b]{0.49\textwidth}
		 \centering
		 \includegraphics[width=\linewidth]{figures/Banos_S1_shuf_MCNN_10_error_check.png}
		 \caption{10 clusters}
	 \end{subfigure}
	\caption{Hyperparameters tuning of \mcnn with first subject of \banosdataset dataset. \TG{It's a bit weird to use accuracy here while F1 score was used before. 
	Couldn't you just use F1 score here too?}}
	\label{fig:mcnn-tuning-error}
\end{figure}

\subsection{\mondrianforest Hyperparameters}

Figure~\ref{fig:mondrian-tuning} shows the impact of the \mondrianforest hyperparameters on
the classification performance. Occasionally, dashed lines are used to
emphasize the minimum and the maximum \TG{That's a comment for the caption. Why ``occasionally'' and not always?}.

The base count hyperparameter (Figure~\ref{fig:mondrian-base-count}) has a
very substantial impact on classification performance; the smallest value
(\TG{add value here}) results in the best performance. On the contrary, the
budget hyperparameter (Figure~\ref{fig:mondrian-budget}) only has a
moderate impact on classification; the best value is slightly below
1.0. Finally, the discount hyperparameter
(Figure~\ref{fig:mondrian-discount}) has a negligible impact on the
performance; the best-performing value is 0.1.

\TG{Add the selected values of hyperparameters somewhere before, maybe in sub-figure captions in F1 scores.}

\begin{figure}
	 \centering
	 \begin{subfigure}[b]{0.49\textwidth}
		\centering
		\includegraphics[width=\textwidth]{figures/Banos_S1_shuf_Mondrian_T10_check.png}
		\caption{Impact of the base count with 10 trees, a budget of $1.0$, and a discount factor of $0.2$. \TG{Could you simplify the figure legend to just show ``Mondrian - <base count>''? 
		it's a bit difficult to read and all parameters are the same and reported in this caption. Same comment for the other sub-figures.}} 
		\label{fig:mondrian-base-count}
	\end{subfigure}
	\hfill
	 \begin{subfigure}[b]{0.49\textwidth}
		 \centering
		 \includegraphics[width=\textwidth]{figures/Banos_S1_shuf_Mondrian_T10_bc_0.1_budget_check.png}
		 \caption{Impact of the budget with 10 trees, a base count of $0.1$, and discount factor of $0.2$.}
		 \label{fig:mondrian-budget}
	 \end{subfigure}
	 \hfill
	 \begin{subfigure}[b]{0.49\textwidth}
		 \centering
		 \includegraphics[width=\textwidth]{figures/Banos_S1_disount_check.png}
		 \caption{Impact of the discount factor with 10 trees, a budget of $1.0$, and a base count of $0.1$. \TG{put this graph on the same 0-1 scale as the two other ones.}}
		 \label{fig:mondrian-discount}
	 \end{subfigure}
		\caption{Hyperparameters tuning for Mondrian with first subject of \banosdataset dataset.}
		\label{fig:mondrian-tuning}
\end{figure}



\section{Conclusion}
In this study, we observed that Naïve Bayes and the Hoeffding Tree with Naïve
Bayes leaves were the best classifiers for Datastream Human Activity
Recognition.  We also discovered that the Hoeffding Tree and the MCNNs were
more resilient to a drift that the others.  We learn that a memory-bound has an
important impact on the classification performances of the Mondrian Forest,
to the point where it cannot learn nor adapt to a drift.  Finally, we found out
that the classifier used does not have an impact on power consumption.

The next step of this study will be to decrease the memory dependency of the
Mondrian Forest and to improve its runtime to make it more competitive with
classifiers such as the Naïve Bayes or the Hoeffding Tree.


\externalbibliography{yes}
%\bibliographystyle{plain}
\bibliography{paper}
\end{document}

% vim: tw=50 ts=2

