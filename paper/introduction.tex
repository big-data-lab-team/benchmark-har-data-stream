\section{Introduction}
\label{sec:introduction}

In this paper, we evaluate the performances of
data stream classification algorithms to provide
insights about which algorithm to choose regarding
the situation.
This evaluation check the classification
performances as well as the resource usage (energy
and memory).

\begin{itemize}
		\item Low-level environment, wearable device
		\item Embedded with OS functions unavailable
		\item Human Activity Recognition.
		\item Expend "regarding the situation" to talk about IoT.
		\item concept drift.
		\item descibre the goal of the data stream
				training (we evaluate with no knowledge of
				the stream). We are looking for a model
				that generalize easily.
\end{itemize}
%Introduce the internet of thing.
%Description of the IoT
%Example of use
%Describe the pipeline
%Explain the drawback of this pipeline
%Privacy issue
%Battery issue
%NOTE: Keep the next three lines :D.
%\cite{sensor-energy-consumption} (conclusion, second paragraph) communication uses more energy.
%\cite{leach} and \cite{sensor-energy-model}(page 3, first column, check equations and values)
%\cite{sensor-network-survey} : "Since the sensor nodes are often inaccessible, the lifetime of a sensor network depends on the lifetime of the power resources of the nodes"

%Maintenance problems (not necesserly related to the pipeline.)

%Outline of the report

%How do we repatriate the data when it has been
%produce on a sensor on the far end of a sensor
%network? How do we do it efficiently?

%How to we mine those data? Which framework?
\subsection{Related Work}
%Implemented in R modules
%List des classifier, quel sont les résultat
%The comparison in, the benchmark done in, the study in
%Profiling
%EN discussion, compared avec les résultat obtenu dans ce papier.
%En future work, éventuellement pousser la
%comparaison pour évaluer les étape de chaque algo en terme mémoire/énergy
The work in \cite{memory_consumption_machine_learning}
explores the memory consumption and
the runtime of many R classifiers. The conclusion proposes different ways to
limit the overhead related to the implementation of the R module.

\cite{Janidarmian_2017} is an extensive study that
observes offline classifiers' performance with
wearable sensor data. This study shows high
accuracy using a K-fold validation and good
accuracy when using subject-independent
cross-validation. It also explores different
sensor placement as well as different window
sizes. The conclusion states that KNN is the most
stable classifier across sensor placement and
window size. Additionally to the classification
performances, the study also analyzes the
trade-off between runtime and efficiency.

\cite{omid_2019} studies the feasibility of
running classifier directly on wearable sensors.
The paper focuses on a comparison between a
Feedforward Neural Network~(FNN) and a K-nearest
neighbor~(KNN). It shows that a trained neural
network achieves high accuracy and performs better
than KNN. The dataset was acquired with the
Neblina, a wearable sensor placed on the right
forearm. The data from the Nebline was merged to
send to the computer a stream of 9-axis. Only
then, the stream was processed to extract features
and feed the classifier.

These three studies compared classification
performances, runtime, and resource usage of many
classifiers. Two of them use wearable sensor data
and one focuses on runtime and memory usage.
However, none of them is applied to data stream
situations because they either rely on K-fold
validation or cross-subject validation.
Additionally, these papers do not compare energy
consumption. The only mention of energy is made in
\cite{omid_2019} where the consumption is assumed to be
constant. Therefore, only the runtime affects 
energy consumption, and reducing the runtime
reduces the energy needed.

%Étendre related work, regarder les papiers qui ont cité ces papier
%En particulier la référence Janidarmian_2017, pareil pour memory_consumption_machine_learning

% vim: tw=50 ts=2
