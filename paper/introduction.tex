\section{Introduction}
\label{sec:introduction}

In this paper, we evaluate the performances of
data stream classification algorithms to provide
insights about which algorithm to choose regarding
the situation.
This evaluation check the classification
performances as well as the resource usage (energy
and memory).

\begin{itemize}
		\item Low-level environment, wearable device
		\item Human Activity Recognition.
		\item Expend "regarding the situation" to talk about IoT.
		\item concept drift.
\end{itemize}
%Introduce the internet of thing.
%Description of the IoT
%Example of use
%Describe the pipeline
%Explain the drawback of this pipeline
%Privacy issue
%Battery issue
%NOTE: Keep the next three lines :D.
%\cite{sensor-energy-consumption} (conclusion, second paragraph) communication uses more energy.
%\cite{leach} and \cite{sensor-energy-model}(page 3, first column, check equations and values)
%\cite{sensor-network-survey} : "Since the sensor nodes are often inaccessible, the lifetime of a sensor network depends on the lifetime of the power resources of the nodes"

%Maintenance problems (not necesserly related to the pipeline.)

%Outline of the report

%How do we repatriate the data when it has been
%produce on a sensor on the far end of a sensor
%network? How do we do it efficiently?

%How to we mine those data? Which framework?
\subsection{Related Work}
\begin{itemize}
	\item Memory consumption~\cite{memory_consumption_machine_learning}
  \item Accuracy and sensors~\cite{Janidarmian_2017}
\end{itemize}
\begin{itemize}
	\item Lister les algo et les expliquer avec une quick description.
\end{itemize}
% vim: tw=50 ts=2
