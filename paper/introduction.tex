\section{Introduction}
\label{sec:introduction}
\TG{The introduction is too basic, if this is going to IEEE IoT, 
we can assume that people know the basic concept of IoT, what is a connected objects and what is a sensor.
I would suggest the following outline for the introduction:
\begin{itemize}
\item what is data stream analytics, difference with online algorithms, what are the main problems
\item definition of the centralized and decentralized pipelines in IoT
\item impact on privacy and energy consumption
\item problem statement: we compare data stream classification algorithms from the point of view of the IoT
\item human activity recognition as an application of the IoT
\end{itemize}
Most of the content is already there, it's a matter of restructuring and updating the tone.
}
The Internet of
Things is a concept in which a wide variety
of objects are connected to the Internet to
provide new services.  The
type of objects that can be connected could be a
watch\footnote{\url{https://www.apple.com/watch/}}
as well as a street light~\cite{smart-lamp-2011}
or a basic sensor. A sensor is a very small device
designed to detect events in its environment such
as temperature, humidity, or acceleration.  The
sensor is said wearable when it is incorporated
into clothing or worn as an implant. These objects often
embed a wireless communication system like wifi
or Bluetooth.  

%Example of use
Connected objects can be used in various
situations.  For instance, wearable sensors that
gather data about human activity, as described
in~\cite{recofit}, can be leveraged to improve
athletes performances and health.  As a concrete
example, the Motsai company develops the Neblina,
a wearable device as small as a coin that
incorporates nine motion sensors with 64KB of main
memory\footnote{\url{http://docs.motsai.com/Neblina/Neblina_Module/V2/Datasheet.html}}.
This wearable sensor is designed to track and
analyze human motion. Due to the limited memory,
the sensor is not embeded with an operating system,
therefore most of the existing code that relies on
operating system functions must be adapted.

The current pipeline that processes the data from
connected appliances is centralized, as
illustrated in~\cite{recofit}.  Data are produced
on a device, then transmitted to a centralized
cloud or a laptop where they are stored and turned
into a proper dataset. Then, the dataset is used
to extract useful information in an offline
manner.  The extraction is qualified as offline
because it starts once all data are collected. For
instance, when training a Machine Learning model,
the data are gathered then carefully curated to select
the best features. Then the model is trained and
tested on the data set. Finally, the model is
transmitted to be used as-is.

This centralized pipeline raises multiple issues.
A privacy issue, because users are increasingly
sensitive about the use made of their personal
data.  A battery issue because the centralized
approach increases information transmission through
the wireless network which is the most
energy-consuming action for a connected
device~\cite{sensor-network-survey,
sensor-energy-model}.

To tackle these privacy and battery issues,
algorithms with small memory footprint were
developed to mine information directly on the
devices rather than in a centralized cloud.  These
algorithms were inspired of Big Data, in
particular, from data stream algorithms because it
is common for these fields to handle more data
than available resources can accommodate.
Designing such method allow transmitting only the
relevant data to the cluster. The survey
in~\cite{kejariwal2015} reviews the problems faced
by connected devices.

A data stream algorithm is designed to process an
infinite sequence of element that cannot be
stored, namely a data stream. A sensor can be
seen as producing data streams since it emits data
for its lifetime, these data are accessible in
sequence, and the storage is too small to save the
entire stream. Data stream classification is
performing a classification task where the dataset
is a data stream. This means that the model
accesses data points in sequence, should be able
to classify at any time, and cannot store all the
points in memory. The task is designated as Human Activity
Recognition when we try to recognize human motion.
Because the stream is infinite, the concept
learned by the model may evolve over time, which is
called a concept drift.

In this paper, we evaluate the learning of data
stream classifiers applied to Human Activity
Recognition (HAR) in order to provide insights
about which algorithm to choose regarding the
situation. The evaluation is done without any
prior knowledge of the data to simulate a brand
new model trying to learn from a sensor. This
study evaluates the classification performances as
well as resource usage (power, runtime, and
memory). In particular, we want to observe any
relation between resources and classification
performances with the final goal being to estimate the
feasibility of running data stream classifier
on a smart device.

%NOTE: Keep the next three lines :D.
%\cite{sensor-energy-consumption} (conclusion, second paragraph) communication uses more energy.
%\cite{leach} and \cite{sensor-energy-model}(page 3, first column, check equations and values)
%\cite{sensor-network-survey} : "Since the sensor nodes are often inaccessible, the lifetime of a sensor network depends on the lifetime of the power resources of the nodes"

\subsection{Related Work}
%Implemented in R modules
%List des classifier, quel sont les résultat
%The comparison in, the benchmark done in, the study in
%Profiling
%EN discussion, compared avec les résultat obtenu dans ce papier.
%En future work, éventuellement pousser la
%comparaison pour évaluer les étape de chaque algo en terme mémoire/énergy
The work in \cite{memory_consumption_machine_learning}
explores the memory consumption and
the runtime of many R classifiers. The conclusion proposes different ways to
limit the overhead related to the implementation of the R module.

\cite{Janidarmian_2017} is an extensive study that
observes offline classifiers' performance with
wearable sensor data. This study shows high
accuracy using a K-fold validation and good
accuracy when using subject-independent
cross-validation. It also explores different
sensor placement as well as different window
sizes. The conclusion states that KNN is the most
stable classifier across sensor placement and
window size. Additionally to the classification
performances, the study also analyzes the
trade-off between runtime and efficiency.

\cite{omid_2019} studies the feasibility of
running classifier directly on wearable sensors.
The paper focuses on a comparison between a
Feedforward Neural Network~(FNN) and a K-nearest
neighbor~(KNN). It shows that a trained neural
network achieves high accuracy and performs better
than KNN. The dataset was acquired with the
Neblina, a wearable sensor placed on the right
forearm. The data from the Nebline was merged to
send to the computer a stream of 9-axis. Only
then, the stream was processed to extract features
and feed the classifier.

These three studies compared classification
performances, runtime, and resource usage of many
classifiers. Two of them use wearable sensor data
and one focuses on runtime and memory usage.
However, none of them is applied to data stream
situations because they either rely on K-fold
validation or cross-subject validation.
Additionally, these papers do not compare energy
consumption. The only mention of energy is made in
\cite{omid_2019} where the consumption is assumed to be
constant. Therefore, only the runtime affects 
energy consumption, and reducing the runtime
reduces the energy needed.

%Étendre related work, regarder les papiers qui ont cité ces papier
%En particulier la référence Janidarmian_2017, pareil pour memory_consumption_machine_learning

% vim: tw=50 ts=2
